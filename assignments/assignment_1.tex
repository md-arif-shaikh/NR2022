\documentclass[10pt]{article}
\usepackage{amsmath}
\usepackage[margin=1in]{geometry}
\usepackage[dvipsnames]{xcolor}
\usepackage[colorlinks=true, urlcolor=Maroon]{hyperref}
\usepackage{xcolor}
\usepackage{libertine}
\usepackage{titling}

\renewcommand{\maketitle}{%
  \begin{flushleft}
    {\scshape \Large \bfseries \thetitle}\\
    {\large Introduction to Numerical Relativity, Spring 2022, ICTS-TIFR}

    \vspace{0.25cm}

    {Md Arif Shaikh, Postdoctoral Fellow\\
      \href{https://www.icts.res.in/}{International Centre for Theoretical Sciences, Tata Institute of Fundamental Research}\\
      Survey No. 151, Shivakote, Hesaraghatta Hobli, Bengaluru -- 560 089, India.\\
      Email: \href{mailto:arif.shaikh@icts.res.in}{arif.shaikh@icts.res.in}, Webpage: \href{https://md-arif-shaikh.github.io/}{md-arif-shaikh.github.io}
      \rule{\textwidth}{1pt}
    }
  \end{flushleft}
}

\newcommand{\R}[1]{{}^{(4)}R_{#1}}
\title{Solution of assignment \#1}
\begin{document}
\maketitle
\begin{enumerate}
\item Using $\tilde{\omega}^a.\boldsymbol{e}_b = \delta^a_b$, $\boldsymbol{e}_a.\boldsymbol{e}_b = g_{ab}$ and $\tilde{\omega}^a.\tilde{\omega}^b = g^{ab}$, prove that $g^{ab}$ is the inverse of $g_{ab}$.
  \begin{align}
    \label{eq:1}
    g_{ab}g^{bc}
    & = (\boldsymbol{e}_a . \boldsymbol{e}_b) (\tilde{\omega}^b . \tilde{\omega}^c)\\
    & = (\boldsymbol{e_a}.\tilde{\omega}^c) (\boldsymbol{e}_b . \tilde{\omega}^b)\\
    & = \delta^c_a
  \end{align}
\item Using the coordinate transformations $\boldsymbol{e}_{a'} = \boldsymbol{e}_{a}M^a_{a'}$, $\tilde{\omega}^{a'} = M^{a'}_{a}\tilde{\omega}^a$ and the inverse matrices $M^{a'}_b = (M^b_{a'})^{-1}$, prove that the components of the vectors, $\boldsymbol{A} = A^ae_a$, dual-vectors, $\tilde{\boldsymbol{B}} = B_b \tilde{\omega}^b$, and mixed tensors, $\boldsymbol{T} = T^a_b e_a \tilde{\omega}^b$, are transformed as $A^{a'} = M^{a'}_a A^a$, $B_{a'} = B_a M^{a}_{a'}$ and $T^{a'}_{b'} = M^{a'}_a T^a_b M^{b}_{b'}$, respectively.

  \begin{align}
    \label{eq:2}
    \boldsymbol{A}
    & = A^{a}e_{a}\\
    & = A^{a}e_{a'} M^{a'}_{a}\\
    & = A^{a}M^{a'}_{a}e_{a'}\\
    & = A^{a'} e_{a'}
  \end{align}
  implies $A^{a'} = A^{a}M^{a'}_{a}$

  \begin{align}
    \label{eq:3}
    \boldsymbol{\tilde{B}}
    & = B_b\tilde{\omega}^b\\
    & = B_b \tilde{\omega}^{b'} M^{b}_{b'}\\
    & = B_b M^{b}_{b'}\tilde{\omega}^{b'}\\
    & = B_{b'}\tilde{\omega}^{b'}
  \end{align}
  implies $B_{a'} = B_{a} M^{a}_{a'}$.
  Finally,
  \begin{align}
    \label{eq:4}
    \boldsymbol{T}
    & = T^{a}_b e_a \tilde{\omega}^b\\
    & = T^{a}_b e_{a'}M^{a'}_{a}\tilde{\omega}^{b'}M^{b}_{b'}\\
    & = T^{a}_{b}M^{a'}_{a}M^{b}_{b'}e_{a'}\tilde{\omega}^{b'}\\
    & = T^{a'}_{b'}e_{a'}\tilde{\omega}^{b'}
  \end{align}
  implying $T^{a'}_{b'} = T^{a}_{b} M^{a'}_{a}M^{b}_{b'}$.
  
\item Using the Bianchi identities,
  \begin{equation}
    \label{eq:bianchi-identities}
    \nabla_e\R{abcd} + \nabla_d\R{abec} + \nabla_c\R{abde} = 0,
  \end{equation}
  prove that $\nabla^aG_{ab} = 0$, where
  \begin{equation}
    \label{eq:Gab}
    G_{ab} = \R{ab} - \frac{1}{2}g_{ab} \R{}
  \end{equation}

  We start by contracting Eq. \eqref{eq:bianchi-identities} with $g^{ac}$ which gives
  \begin{equation}
    \label{eq:3-1}
    \nabla_{e}\R{bd} - \nabla_d \R{be} + \nabla^a\R{abde} = 0
  \end{equation}
  where we have used $\R{abcd} = - \R{abdc}$ in the second term. Now we contract Eq. \eqref{eq:3-1} with $g^{eb}$ to get
  \begin{equation}
    \label{eq:}
    2 \nabla^b \R{bd} - \nabla_d \R{} = 0 \Rightarrow \nabla^a\R{ab} - \frac{1}{2} \nabla^a (g_{ab}\R{}) = 0 \Rightarrow \nabla^a G_{ab} = 0
  \end{equation}

\item Derive the curvature invariants $R^{ab}R_{ab}$ and $R$ for Schwarzschild and Kerr spacetime in the vacuum case.

  Einstein's Field equation is given by
  \begin{equation}
    \label{eq:EFE}
    R_{ab} + \frac{1}{2} g_{ab}R = 0
  \end{equation}
  contracting with $g^{ab}$ gives
  \begin{equation}
    \label{eq:}
    g^{ab}R_{ab} + \frac{1}{2} g^{ab}g_{ab}R = 0 \Rightarrow R + R = 0 \Rightarrow R = 0
  \end{equation}
  From which it automatically follows that $R^{ab}R_{ab} = 0$.

\item Exercise 2.1 of Baumgarte and Shapiro.
\item Exercise 2.12-2.14 of Baumgarte and Shapiro.
  \begin{enumerate}
  \item The acceleration is given by
    \begin{equation}
      \label{eq:acceleration}
      a_a = n^b \nabla_b n_a
    \end{equation}
    Component along $n^a$ is given by
    \begin{equation}
      \label{eq:projection-along-na}
      n^aa_a = n^a n^b \nabla_b n_a = \frac{1}{2}  n^b \nabla_b(n^an_a) = 0
    \end{equation}

  \item
  \item 
  \end{enumerate}
  
\item Exercise 2.34 of Baumgarte and Shapiro.
\end{enumerate}
\end{document}
